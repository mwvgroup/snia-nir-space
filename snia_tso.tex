\title{Cosmological Transients from LSST: Rapid Response with UV, Optical, and NIR Follow-Up}

\section{Type Ia Supernova Progenitors and Cosmology with Rapid UV-Optical-NIR Followup}

LSST will obtain good lightcurves for 100,000~SNeIa over 10 years.  These will allow for measurements of the expansion of the Universe out to z~0.8 based on restframe optical observations.  Achieving the most accurate and precise constraints on dark energy will require improved understanding of SN~Ia behavior across cosmic time and robust, well-calibrated luminosity distances using both restframe optical+infrared.

\subsection{SN~Ia Cosmology}

Type Ia supernovae are intrinsically less variable in the restframe NIR, in particular through $H$-band ($1.5--1.8$~\mu) \citep{Krisciunas, Wood-Vasey, CSP} and suffer less uncertainty due to the effects of dust in their host galaxy and our own Milky Way.

Photometric calibration remains a key limitation in supernova cosmology\citep[c.f.][]{Scolnic}.  Achieving consistent relative calibration from the optical through to the NIR to 2\% will be key to making accurate and precise measurements of luminosity distance that will allow for the measurement of the equation of state parameter of dark energy, $\omega$, and any potential variation with redshift.  Close coordination between LSST and spaced-based observations both in observed samples of supernova and calibration of their overall systems is vital toward maximizing the science potential of SN~Ia cosmology in the decade to come. 

\citet{Astier16} presented a clear case for joint LSST+Euclid observations.  We here argue for the addition of spectroscopic observations as (i) key indicators of SN~Ia populations and potential variation, and (ii) an increase in wavelength coverage to $\lambda\sim4\mu$m to allow for rest-frame H-band observations.

\subsection{SN~Ia Progenitor Systems}

The lack of knowledge about SN~Ia progenitor systems casts a shadow over treating SNeIa as standardizable across 8~Gyr ($z\sim1$) of cosmic history.  Efforts to identify progenitor systems for detected SNeIa have so far come up with nothing, and very constraining limits in the case of the two closest SNeIa of the past decade (?) SN~2011fe \citep{??} and SN~2014J \citep{??} strongly suggest we may never independently observe a progenitor system.

Two key avenues for understanding progenitor systems are 
(i) obtaining specific data from the explosion from very early-time measurements (starting at <1 hour) that illuminates the system.
(ii) obtaining a full set of observations in time and wavelength that capture the entirety of the energy from originally created Ni$^{56}$.

\subsubsection{Early-Time Flash Frames of the Progenitor System}
If observed within hours on a short time scale, the light from the explosion can illuminate key features of the progenitor system.  The three key progenitor system features would be 
\begin{itemize}
\item Any accretion disk+stream from a non-degenerate companion onto a degenerate companion.
\item The shadow of the companion itself.
\item Ejected mass from the system.
\end{itemize}

The most boring answer of ``nothing'' would indicate that the progenitor system was very compact, likely from two degenerate objects.  

\subsubsection{Bolometric SED: All the photons, All the energy, All the elements}
The amount of Ni$^{56}$ produced in a SN~Ia explosion is a fundamental prediction of the SN~Ia explosion models, while the elements produced reveal the details of the explosion dynamics, in particular the transition from a deflagration to a detonation in the propagation of the nuclear burning front.

SN~Ia emission is line-blanketed through the expanding photosphere so severely that there's no significant emission blueward of the UV.  At least 95\% (?) of the flux comes out between the UV and 5~$\mu$m.  A bolometric SED from $0.3\mu$m$--5.0\mu$m thus captures almost all of the EM emission giving a full accounting of both the energetic and elemental signatures.
